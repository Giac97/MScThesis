\documentclass[11pt,a4paper]{article}
\usepackage[utf8]{inputenc}
\usepackage[english]{babel}
\usepackage{amsmath}
\usepackage{amsfonts}
\usepackage{amssymb}
\usepackage{graphicx}
\usepackage{hyperref}
\usepackage[left=2cm,right=2cm,top=2cm,bottom=2cm]{geometry}
\author{Giacomo Becatti}
\title{Nanoparticle Assembled Thin Films}
\begin{document}
\maketitle
\begin{abstract}
Thin films assembled from nanoparticles deposited on a substrate in such a way that a certain degree of individuality of the individual clusters is mantained, present interesting and peculariar properties ranging from structural to electrical and thermal. The high number of grains and grain-grain boundary, the porosity and other structural characteristic consequence of the low energy deposition of the nanoparticles result in complex electrical and thermal conductivity properties.
\end{abstract}
\section{Introduction}
While films assembled starting from nanopartilces can be thought as the very first application of nanotechnlogy, showing up as  in vases dating back to medieval Europe \cite{heterogeneous_borgia_2002}, even in recent times they were thought have conduction properties similar to "classical" thin films assembled starting from single atoms and molecules \cite{electrical_mirigliano_2021}. It was however found that, owing to their much more complex geometrical structure, that their electrical \cite{nonohmic_mirigliano_2019}\cite{electrical_mirigliano_2021} conduction properties follow a more complex, non-ohmic law. Such complex conductive was found could be exploited in the making of components used in neuromorphic computing \cite{binary_mirigliano_2021}.
\subsection{Neuromorphic Computing}
With the slowing down of Moore's law new unconventional approaches to computing, ranging from quantum to biological computing, are emerging and getting more and more attention. Among such approaches, which fall under the umbrella term of unconventional computing, one, known as neuromorphic computing, aims at reproducing the behaviour of synapses in the human brain at the hardware level.

\bibliographystyle{plain}
\bibliography{bibliography}
\end{document}